


\documentclass[11pt, oneside]{article}   	% use "amsart" instead of "article" for AMSLaTeX format
\usepackage{geometry}                		% See geometry.pdf to learn the layout options. There are lots.
\geometry{a4paper}                   		% ... or a4paper or a5paper or ... 
%\geometry{landscape}                		% Activate for rotated page geometry
%\usepackage[parfill]{parskip}    		% Activate to begin paragraphs with an empty line rather than an indent
\usepackage{graphicx}				% Use pdf, png, jpg, or eps§ with pdflatex; use eps in DVI mode
								% TeX will automatically convert eps --> pdf in pdflatex		
\usepackage{amsmath}
\usepackage{amssymb}

%SetFonts

%SetFonts


\title{The Lax-Wendroff Scheme}
\author{Jo\~ao Bettencourt}
\date{February 2023}							% Activate to display a given date or no date

\begin{document}
\maketitle
%\section{}
%\subsection{}

The Lax-Wendroff scheme starts from the linear 1-d advection equation:
\begin{equation}\label{eqn:1dlinadve} % requires amsmath; align* for no eq. number
   \frac{\partial u}{\partial t} = -c  \frac{\partial u}{\partial x},
\end{equation}

and uses the Taylor series expansion of $u(x,t+\Delta t)=u(x_m,t^{n+1})=u_{m}^{n+1}$:
\begin{equation}\label{eqn:taylor}
   u_{m}^{n+1} = u_{m}^{n} + \Delta t \frac{\partial u}{\partial t} + \frac{\Delta t^2}{2} \frac{\partial^2 u}{\partial t^2}+O(\Delta t^3).
\end{equation} 

If we replace the time derivatives in (\ref{eqn:taylor}) by the spatial derivative from (\ref{eqn:1dlinadve}), 
we get\footnote{Note that from (\ref{eqn:1dlinadve}) we have:
\(\frac{\partial^2 u}{\partial t^2} = -c  \frac{\partial}{\partial t} \left(\frac{\partial u}{\partial x}\right)=
-c  \frac{\partial }{\partial x} \left(\frac{\partial u}{\partial t}\right)=-c  \frac{\partial }{\partial x} \left(-c\frac{\partial u}{\partial x}\right)=
c^2  \frac{\partial^2 u}{\partial x^2}\)}: 
\begin{equation}\label{eqn:taylor2}
   u_{m}^{n+1} = u_{m}^{n} + \Delta t (-c\frac{\partial u}{\partial x}) + \frac{\Delta t^2}{2} c^{2}\frac{\partial^2 u}{\partial x^2}+O(\Delta t^3).
\end{equation} 

Now we can approximate the spatial derivatives with centred differences to obtain the Lax-Wendroff scheme:
\begin{equation}\label{eqn:lax}
   u_{m}^{n+1} = u_{m}^{n} - c\frac{\Delta t}{2\Delta x}(u_{m+1}^{n}-u_{m-1}^{n}) + c^2\frac{\Delta t^2}{\Delta x^2}(u_{m+1}^{n}-2u_{m}^{n}+u_{m-1}^{n}),
\end{equation} 

where we recognize that the last term in (\ref{eqn:lax}) is a dissipative term added to (\ref{eqn:1dlinadve}) to control what would otherwise be an unstable Euler method. 
To find out the order of the Lax-Wendroff scheme, we can return to (\ref{eqn:taylor2}) and rewrite it as:
\begin{equation*}\label{eqn:taylor3}
   \frac{u_{m}^{n+1} - u_{m}^{n}}{ \Delta t } =-c\frac{\partial u}{\partial x} + \frac{\Delta t}{2} c^{2}\frac{\partial^2 u}{\partial x^2}+O(\Delta t^2),
\end{equation*} 
where we then replace the spatial derivatives with their second-order approximations:
\begin{multline*}\label{eqn:taylor4}
   \frac{u_{m}^{n+1} - u_{m}^{n}}{ \Delta t } =-c \left[ \frac{u_{m+1}^{n}-u_{m-1}^{n}}{2\Delta x}   + O(\Delta x^2)\right] + \\
  +\frac{\Delta t}{2} c^{2}\left[ \frac{u_{m+1}^{n}-2*u_{m}^{n} - u_{m-1}^{n}}{\Delta x^2}  + O(\Delta x^2)\right]+O(\Delta t^2),
\end{multline*} 

to find that the scheme is second order in space as in time.

\end{document} 